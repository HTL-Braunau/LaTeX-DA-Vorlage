%!TEX root=../Vorlage_DA.tex
%	########################################################
% 					Standards für Technikerprojekte
%	########################################################


%	--------------------------------------------------------
% 	Überschrift, Inhaltsverzeichnis
%	--------------------------------------------------------
\chapter{Standards für Technikerprojekte}

(Aus Rundschreiben Nr. 60 der BMBWK aus 1999 unter der Geschäftszahl 17.600/101-II/2b/99)  

%	--------------------------------------------------------
% 	Definition eines Technikerprojektes
%	--------------------------------------------------------
\section{Definition eines Technikerprojektes}

Ein Technikerprojekt versteht sich als abschließender Leistungsnachweis des gesamten Ausbildungsweges an einer mittleren technisch-gewerblichen Schule (Fachschule). Es soll dem Schüler in praxisnaher Form Gelegenheit zur Umsetzung und Vertiefung der in seiner Ausbildungszeit erworbenen Fachkenntnisse und Fertigkeiten geben.

Ein Technikerprojekt ist eine von einem zwei- bis sechsköpfigen Schülerteam durchzuführende, in sich geschlossene Arbeit. Das Team steht dabei unter der Leitung eines hauptverantwortlichen Projektbetreuers, der ein Lehrer mit entsprechender Fachexpertise sein muss. Die Aufgabenstellung soll industriespezifischen oder gewerblichen Charakter haben und die Durchführung möglichst in Kooperation mit einem außerschulischen Partner erfolgen. Die Dauer eines Technikerprojektes beträgt mindestens 3 Monate während des letzten Ausbildungsjahres. Neben den fachlichen Aufgaben und Analysen sollen umweltrelevante Fragestellungen sowie Aspekte der Wirtschaftlichkeit miteingeschlossen werden. Integrierter Bestandteil eines Technikerprojekts ist eine Dokumentation und eine gut vorbereitete Präsentation, die sich moderner Technologien zur Veranschaulichung bedienen soll.

%	--------------------------------------------------------
% 	Das pädagogische Konzept
%	--------------------------------------------------------
\section{Das pädagogische Konzept}

Das pädagogische Konzept orientiert sich an Prinzipien, die in zwei Gruppen zusammengefasst werden können:

\begin{itemize}
	\item Die inhaltlichen Grundsätze orientieren sich am im Einzelfalle höchstmöglich erreichbaren Maß an Praxisnähe. Technikerprojekte definieren sich dabei über praktische Problemstellungen, sorgfältige Planung, begleitendes Projektmanagement, eine Dokumentation, die Einbindung moderner Präsentationstechniken sowie der Beachtung der Grundsätze der Qualitätssicherung.
	\item Im Bereich der Persönlichkeitsbildung werden in Ergänzung und Vertiefung zu den allgemeinen Bildungszielen die Schulung der Teamfähigkeit, die individuelle Förderung spezieller Begabungen, die intensive Erfahrung von Selbständigkeit und Eigenverantwortlichkeit, die Stärkung des Selbstbewusstseins und die Freiwilligkeit der Arbeitsleistung in den Mittelpunkt gestellt.
	
\end{itemize}


%	--------------------------------------------------------
% 	Didaktische Konsequenzen
%	--------------------------------------------------------
\section{Didaktische Konsequenzen}

Das Erreichen dieser Ziele erfordert in weiten Feldern eine Neugewichtung der Unterrichtsprinzipien. So werden das Prinzip des gegenstandsübergreifenden Unterrichts, \glqq Team-teaching\grqq (insbesondere auch durch Lehrer verschiedener Fächergruppen), schülerzentrierter Unterricht, zielorientiertes Arbeiten, die Entwicklung eines Zeit- und Kostenbewusstseins sowie Methodenvielfalt der Wissensaneignung in dieser Phase des Unterrichtsgeschehens betont werden.


%	--------------------------------------------------------
% 	Das Ziel: Eine neue Qualität in der Ausbildung
%	--------------------------------------------------------
\section{Das Ziel: Eine neue Qualität in der Ausbildung}

Die  Durchführung eines Technikerprojekts hat das Ziel dem einzelnen Schüler  

\begin{itemize}
	\item Fachkompetenz
	\item Methodenkompetenz
	\item Sozialkompetenz und
	\item Selbstkompetenz
\end{itemize}



%	--------------------------------------------------------
% 	Entstehungs- und Entscheidungsphase
%	--------------------------------------------------------
\section{Entstehungs- und Entscheidungsphase}

Schulexterne Kontakte sind bereits im Vorfeld anzustreben. Projekte mit außerschulischen Partnern sind das primäre Ziel, werden aber nicht immer realisierbar sein. Bei rein innerschulischen Projekten sind vorzüglich solche mit schulischer Wertschöpfung anzustreben.

Themenstellungen sollen möglichst gegenstandsübergreifend erfolgen, um dem Schüler ein Höchstmaß an Lösungskompetenz für die Berufspraxis zu vermitteln. Die Projektthemen müssen einen Realitätsbezug zum Berufsfeld des Fachbereiches aufweisen. Es muss gewährleistet sein, dass relevante Kompetenzen aus dem angestrebten Berufsfeld vertieft und umgesetzt werden. Die engere Themenwahl sollte sich dabei möglichst am realen Bedarf in Wirtschaft und Gesellschaft orientieren.

Jedes in die engere Wahl kommende Projekt muss im Interesse eines erfolgreichen Abschlusses ernsthaften Machbarkeitsüberlegungen unterzogen werden. Neben diesen grundsätzlichen Machbarkeitsüberlegungen ist auch die Durchführbarkeit der einzelnen Projektvorschläge unter den gegebenen Rahmenbedingungen gewissenhaft und aufrichtig zu prüfen. Ziel dieser Prüfung ist, dass letztlich jedes begonnene Projekt für den Schüler auf Grund seiner Vorbildung bewältigbar und mit den zur Verfügung stehenden Ressourcen auch durchführbar ist.


%	--------------------------------------------------------
% 	Vorbereitungsphase
%	--------------------------------------------------------
\section{Vorbereitungsphase}

Am Beginn steht die Bildung des Projektteams. Ein solches besteht aus 2 bis 6 Schülern und aus einem (oder mehreren) projektbetreuenden Lehrer, der über die notwendige Fachexpertise verfügen muss. Die Zusammenstellung des Teams kann nach verschiedenen Gesichtspunkten wie etwa Schülerselbstbestimmung, lehrergesteuert, problemorientiert oder auch nach Zufallsaspekten erfolgen. Wenn ein Projekt von mehreren Lehrern betreut wird, ist ein hauptverantwortlicher Projektbetreuer zu nennen.

Die Rahmenbedingungen (rechtliche Fragen, Normen, einschlägige Vorschriften \ldots) sind in das Thema einzuarbeiten und in die Projektdokumentation aufzunehmen.

Ebenso haben Recherchen zum Projektthema und dem fachlichem Umfeld durch das Projektteam in angemessenem Umfang zu erfolgen.


%	--------------------------------------------------------
% 	Durchführungsphase
%	--------------------------------------------------------
\section{Durchführungsphase}

Die Durchführung des Projektes hat in Teamarbeit zu erfolgen, arbeitsteilige Kooperation ist das zentrale Lernziel. Jedem Mitglied des Projektteams sind dabei persönliche Arbeitsanteile klar zuzuordnen, die auch eine individuelle Beurteilung im Rahmen der Teamarbeit erlauben.

Als erste Arbeit sind die Projektziele zu definieren. Anschließend ist ein Projektplan zu erstellen, welcher ein Pflichtenheft, eine Terminplanung, eine Kostenrechnung und die Planung der Einsatzmittel zu enthalten hat.

Die Durchführung des Projektes hat unter Einbeziehung und Nutzung moderner Technologien zu erfolgen. Die genaue Führung eines Projekttagebuches ist unabdingbar.


%	--------------------------------------------------------
% 	Abschlussphase
%	--------------------------------------------------------
\section{Abschlussphase und Projektdokumentation}

Ein wesentlicher Teil des Projektes ist eine Dokumentation, die das Projekt in allen Phasen und Ergebnissen beschreibt. Als Grundlage für diese Dokumentation ist das Projekttagebuch heranzuziehen.

Das zu Beginn erstellte Pflichtenheft ist zwingender Bestandteil der Dokumentation.
Empfohlen wird eine gebundene Dokumentation in mehrfacher Ausführung (Schule-Schüler-ggf. außerschulischer Projektpartner). Die Schule sollte ein Exemplar zu Archivzwecken vorsehen. 


%	--------------------------------------------------------
% 	Projektpräsentation
%	--------------------------------------------------------
\section{Projektpräsentation}

Jedes Mitglied des präsentierenden Projektteams hat sich dabei auf gut vorbereitete  Präsentationsunterlagen zu stützen, wobei möglichst eine freie Rede anzustreben ist. Ebenso ist darauf zu achten, dass jedem Gruppenmitglied möglichst gleicher Zeitanteil bei der Präsentation zukommt.

%	--------------------------------------------------------
% 	Qualitätssichernde Maßnahmen
%	--------------------------------------------------------
\section{Qualitätssichernde Maßnahmen}

Die Vermittlung und Umsetzung der grundlegenden Konzepte der Qualitätssicherung sind integrierender Bestandteil jedes Technikerprojektes. Die Beachtung der qualitätssichernden Maßnahmen (insbesondere der hier festgelegten Standards) wird dabei im Normalfall einem nicht in das Projekt involvierten Experten zukommen.

\textbf{Produkte jeglicher Art haben sich am Stand der Technik zu orientieren. Geräte, Vorrichtungen und Anlagen müssen den geltenden Sicherheitsstandards entsprechen.}
