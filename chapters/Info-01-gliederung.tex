% !TEX root = ../Vorlage_DA.tex

%	########################################################
% 			INFO: Zitieren, Abbildungen, Listing
%	########################################################


%	--------------------------------------------------------
% 	Überschrift, Inhaltsverzeichnis
%	--------------------------------------------------------
\chapter{INFO: Gliederung und Inhalt}

%	--------------------------------------------------------
\section{Gliederung}
%	--------------------------------------------------------

Die vorhergehenden Kapitel sind Muss-Bestandteile der Diplomarbeit.
Ab hier kann die Gliederung (Aufteilung in Kapitel) frei gewählt werden.

Das Dokument soll durch Kapitel und Unterkapitel übersichtlich gegliedert sein. 
Jedes Kapitel bekommt eine Überschrift mit Nummerierung (1.1, 2.2.1, ...). Mehr als 3 Überschriftebenen (1.1.1) sollten vermieden werden. Zusätzlich muss bei jedem Kapitel oder Unterkapitel mit einer Fußnote vermerkt werden, wer diesen Abschnitt erstellt hat.

Umfangreichere Arbeitsergebnisse wie Schaltpläne, Messprotokolle, Datenblätter und das Projekttagebuch kommen in einen Anhang am Ende des Dokuments.

In einem Literaturverzeichnis sind alle verwendeten Quellen und Zitate zu sammeln.

Wird die Diplomarbeit von \textbf{mehreren Personen} gemeinsam erstellt muss erkenntlich gemacht werden wer für welches Kapitel verantwortlich war.

%	--------------------------------------------------------
\section{Beispiel Gliederung}
%	--------------------------------------------------------

Hilfestellung für eine mögliche Benennung und Gliederung der Hauptkapitel:

\begin{description}
\item[Problemanalyse und Spezifikation]
\ \\Erläuterung des \underline{Was}: Aufgabenstellung ganz detailliert (=Pflichtenheft). Hier wird erläutert, was zu machen war. Das Wie ist hier normalerweise fehl am Platz.
\item[Entwurf]
\ \\Erläuterung des Wie: Technologie mit Begründung, bzw. Abwägen der Vor- und Nachteile. Lösungswege, Algorithmen.
\item[Implementierung]
\ \\Zeigt genau die Umsetzung des Entwurfs anhand wesentlicher Quelltext-Ausschnitte.
\item[Test und Inbetriebnahme]
\ \\Erkläre wie getestet wurde und was notwendig ist um das Produkt von Null weg zu installieren.
\item[Bedienungsanleitung]
\ \\Erklärt dem Benutzer die wichtigsten Schritte bei der Bedienung des Systems.
Eventuell ist eine eigenes Bedienerhandbuch für spezielle Benutzergruppen (z.B. Administratoren) notwendig.
\item[Fazit, Schlussfolgerungen]
\ \\Hier werden die Projektergebnisse zusammengefasst. Was ist gelungen was nicht. Welche Erkenntnisse wurden gewonnen.
\item[Persönliche Erfahrungen]
\ \\Hier (und nur hier) darf subjektiv aus der Ich Perspektive über das Projekt philosophiert werden.
\end{description}


%	--------------------------------------------------------
\section{Inhalt}
%	--------------------------------------------------------

Das Ziel ist die eigene Arbeit anderen (technisch versierten, aber projektfremden) Personen nachvollziehbar zu machen.
Ein Mitschüler der ein gutes Informatik Fachwissen hat soll den Text verstehen können.

Die Dokumentation eines Informatik-Projekts soll \textbf{Programmquelltext} enthalten!
Dieser soll gut dokumentiert und lesbar sein. 
Nur wenig zusätzlicher Text soll notwendig sein um das Programm zu verstehen.
Wählt nur jene Programmteile aus, die wirklich interessant sind, nehmt nicht jene Dinge die man in jedem Buch nachlesen kann.

Werden zur Implementierung Libraries, Frameworks, Methoden, etc. verwendet die wesentlich über den normalen Unterrichtsstoff hinausgehen, so sollten diese kurz erklärt werden.
Es ist aber nicht notwendig und nicht zielführend ganze Tutorials zu erstellen.
Nach einer allgemeinen Übersicht, die zu einem grundsatzlichen Verständnis verhelfen soll, genügt es auf entsprechende Quellen zu verweisen.

Der \textbf{Umfang} der Arbeit ist nicht wesentlich, wichtig ist die Qualität des Inhalts.
Ca.\ 40 Seiten pro Person sind eine gute Richtlinie.

Der Inhalt soll aus der \textbf{eigenen Feder} stammen. 
Kopieren fremder Quellen (auch wenn diese ins Deutsche übersetzt werden müssen) ist auf ein Minimum zu beschränken. 
Wenn kopiert wird dann ist immer genau anzugeben von wo. 
Siehe auch Kapitel \ref{ref:zitieren}.

Von Dir verfasste Texte sind Deine Visitenkarte. 
Bemühe Dich alles so gut zu machen wie Du nur kannst.
\begin{itemize}
\item
Strebe nach \textbf{Perfektion} --- auch was die Rechtschreibung und Grammatik angeht.
\item
Gib keinen Text aus der Hand mit dem Du nicht 100\% zufrieden bist.
\item
Lösche unfertige Textstellen ehe Du den Text weitergibst.
\item
Suche Dir jemanden zum Korrekturlesen, in eigenen Texten übersieht man gerne Fehler die einem Anderen sofort auffallen.
\end{itemize}

Keine Erklärungen aus der \textbf{Ich-Perspektive} abgeben (Ausnahme: Fazit am Ende).

Vermeide den Text so zu schreiben wie man spricht, bei Text gelten etwas andere Regeln als bei einer Präsentation.
Verzichte auf das übernehmen mundartlicher Ausdrucksweisen.

Überlege Dir beim Schreiben einer Textstelle ob der Leser das notwendige Hintergrundwissen hat um zu verstehen was Du ausdrücken willst.
Hast Du schon vorher erklärt was man an dieser Stelle wissen sollte?
Versteht man von was die Rede ist?
Beachte, dass Du top in das Thema eingearbeitet bist. 
Was Dir völlig klar erscheint ist dem Leser vielleicht nur ein spanisches Dorf.

Sei aber auch nicht zu weitschweifig. 
Ein guter Text ist kein langer Text sondern ein Text an dem man beim besten Gewissen nichts mehr wegkürzen kann.
Schreibe zuerst etwas weitschweifiger und kürze dann radikal.
Sei nicht zimperlich, wenn Dir eine Stelle nicht gefällt, lösche diese und fange von vorne an.

